%%%%%%%%%%%%%%%%%%%%%%%%%%%%%%%%%%%%%%%%%
% "ModernCV" CV and Cover Letter
% LaTeX Template
% Version 1.3 (29/10/16)
%
% This template has been downloaded from:
% http://www.LaTeXTemplates.com
%
% Original author:
% Xavier Danaux (xdanaux@gmail.com) with modifications by:
% Vel (vel@latextemplates.com)
%
% License:
% CC BY-NC-SA 3.0 (http://creativecommons.org/licenses/by-nc-sa/3.0/)
%
% Important note:
% This template requires the moderncv.cls and .sty files to be in the same 
% directory as this .tex file. These files provide the resume style and themes 
% used for structuring the document.
%
%%%%%%%%%%%%%%%%%%%%%%%%%%%%%%%%%%%%%%%%%

%----------------------------------------------------------------------------------------
%	PACKAGES AND OTHER DOCUMENT CONFIGURATIONS
%----------------------------------------------------------------------------------------

\documentclass[11pt,a4paper,sans]{moderncv} % Font sizes: 10, 11, or 12; paper sizes: a4paper, letterpaper, a5paper, legalpaper, executivepaper or landscape; font families: sans or roman

\moderncvstyle{classic} % CV theme - options include: 'casual' (default), 'classic', 'oldstyle' and 'banking'
\moderncvcolor{green} % CV color - options include: 'blue' (default), 'orange', 'green', 'red', 'purple', 'grey' and 'black'

\usepackage{multicol}


\usepackage[scale=0.75, bottom=0.75cm, top=1cm]{geometry} % Reduce document margins to fit content (vary as needed!)
%\setlength{\hintscolumnwidth}{3cm} % Uncomment to change the width of the dates column
%\setlength{\makecvtitlenamewidth}{10cm} % For the 'classic' style, uncomment to adjust the width of the space allocated to your name

\newcommand{\mydate}{\the\day.\the\month.\the\year}

%----------------------------------------------------------------------------------------
%	NAME AND CONTACT INFORMATION SECTION
%----------------------------------------------------------------------------------------

\input{personaldata.tex}

\firstname{Marvin} % Your first name
\familyname{März} % Your last name
\title{B.Sc. Informatik}

\address{\personaladdressfirst}{\personaladdresssecond}
\phone{\personalphone}
\email{\personalemail}
\extrainfo{\url{https://github.com/marvinmaerz}}



%\homepage{staff.org.edu/~jsmith}{staff.org.edu/$\sim$jsmith} % The first argument is the url for the clickable link, the second argument is the url displayed in the template - this allows special characters to be displayed such as the tilde in this example
%\extrainfo{additional information}
%\photo[70pt][0.4pt]{pictures/picture} % The first bracket is the picture height, the second is the thickness of the frame around the picture (0pt for no frame)
%\quote{"A witty and playful quotation" - John Smith}

%----------------------------------------------------------------------------------------

\begin{document}

%----------------------------------------------------------------------------------------
%	CURRICULUM VITAE
%----------------------------------------------------------------------------------------

\makecvtitle % Print the CV title

%----------------------------------------------------------------------------------------
%	PERSONAL DATA
%----------------------------------------------------------------------------------------
\section{Persönliche Daten}

\begin{multicols}{2}
	\cvitem{Name}{März}
	\cvitem{Vorname}{Marvin}
	\cvitem{Geburtsdatum}{29.07.2003}
	\columnbreak
	\setlength{\hintscolumnwidth}{3cm}
	\cvitem{Geburtsort}{Nürnberg}
	\cvitem[3cm]{Staatsangehörigkeit}{Deutsch}
\end{multicols}


%----------------------------------------------------------------------------------------
%	EDUCATION
%----------------------------------------------------------------------------------------
\section{Bildung}

\cventry{2025--heute}{Masterstudium Informatik}{Friedrich-Alexander-Universität Erlangen-Nürnberg}{Erlangen}{Aktueller Notenschnitt: 1,6}{Abschluss voraussichtlich im Sommer 2026}
%\cventry{2009--2013}{Grundschulabschluss}{Grundschule Neunkirchen}{Neunkirchen am Brand}{}{}

\cventry{2021--2025}{Bachelor of Science (B.Sc.) Informatik}{Friedrich-Alexander-Universität Erlangen-Nürnberg}{Erlangen}{Absolviert mit Note: 2,4}{}

\cventry{2013--2021}{Abitur}{Emil-von-Behring-Gymnasium}{Spardorf}{Note: 2,1}{}



%----------------------------------------------------------------------------------------
%	BACHELOR THESIS
%----------------------------------------------------------------------------------------
\section{Bachelorarbeit (Note 1,3)}

\cvitem{Titel}{\emph{CacheVenture - Guardians of the Memory: Design Proposal of a Cache Simulation Game for Higher Education}}

%\cvitem{Betreuer}{Tobias Baumeister, M.Sc., Lehrstuhl für Informatik 3 (Rechnerarchitektur)}
\cvitem{Beschreibung}{Design, Entwicklung und Evaluation eines interaktiven Cache Simulations-Spiels \comment{unter Einsatz von Gamification }zur Unterstützung der Hochschullehre im Bereich Rechnerarchitektur.}

\cvitem{Verwendete Technologien}{Implementiert mit \textbf{GDScript} (ähnlich zu Python) in Godot, \textbf{git / GitHub} zur Versionskontrolle, \comment{akademisches Schreiben mit}\LaTeX, Projektorganisation via Kanban-Board}

\cvitem{Repository}{\url{https://github.com/marvinmaerz/CacheVenture}}



%----------------------------------------------------------------------------------------
%	ACADEMIC SKILLS
%----------------------------------------------------------------------------------------
\section{Skills und Fächer \normalsize{(Noten in Klammern)}}

\cvitem{Software Engineering}{\textit{Advanced Design and Programming}: \textbf{TypeScript}, industrienahe Programmierparadigmen; \textit{Applied Software Engineering}: Requirements, Design, Testing \& Agile; CacheVenture: Full-Stack-Entwicklung eines spielbasierten Lernsystems}

\cvitem{AI \& Machine Learning}{Implementierung eines eigenen Deep-Learning-Frameworks in \textbf{Python/Numpy} (GitHub)\comment{\textit{Deep Learning}: Programmierung eines Frameworks mit \textbf{Python} und Numpy (siehe GitHub), Übung mit \textbf{PyTorch}}, \textit{Künstliche Intelligenz I \& II} (2,0 \& 2,3): Umfassende Grundlagen und Methoden von symbolischer bis statistischer/sub-symbolischer KI}

\cvitem{Medical AI}{Nebenfach im Master: \textit{AI in Biomedical Engineering}, \textit{AI in Medical Robotics}}

\cvitem{Robotics \& Vision}{\textit{Praktikum Lego Mindstorms} (1,0): Hardware-Software Co-Design eines Lego-Roboters zur automatischen Lösung eines Rubik's Cube (\textbf{Java}, Teamgröße: 4)}

%\cvitem{Data Types}{\textit{Algorithmen und Datenstrukturen} (2,7): \textbf{Java} und Grundlagen der Programmierung}

\cvitem{Cybersecurity}{\textit{Applied IT-Security} (1,3), \textit{Software Exploitation} (1,3)}

\cvitem{Embedded \& Communication Systems}{\textit{Seminar Konzepte von Betriebssystem-Komponenten} (1,0): Paper \& Vortrag über Bluetooth Low Energy in drahtlosen Systemen mit intermittierender Energieversorgung, \textit{Fahrzeugkommunikation} (1,0)}

\cvitem{Computer Architecture}{\textit{Rechnerarchitektur} (1,7): HPC Benchmarking und Optimierung (Speicherlayout, SIMD, Parallelisierung) in \textbf{C} und \textbf{CUDA}}



%\cvitem{Ökonomie}{Nebenfach im Bachelor: \textit{Einführung in die Volkswirtschaftslehre} (1,3), \textit{Mikroökonomie} (2,0)}

%\cvitem{Teamwork}{Zusammenarbeit in interdisziplinären Teams, Versionierung über Git}



% LANGUAGES & INTERESTS SIDE BY SIDE: multicol to save space
\pagebreak
\begin{multicols}{2}
%----------------------------------------------------------------------------------------
%	INTERESTS SECTION
%----------------------------------------------------------------------------------------
\section{Interessen}

\cvlistitem{EDM Musikproduktion \& DJing}

\cvlistitem{Fotografie \& Bildbearbeitung}

\cvlistitem{Programmierung \& Automatisierung}

\cvitem{2018 -- 2025}{Mitglied der FFW Ermreuth-Rödlas}

%\cvlistdoubleitem{Musikproduktion}{Krafttraining}
%\cvlistdoubleitem{Fotografie}{Videospiele}
%\cvlistdoubleitem{Programmieren}{Lesen}

\columnbreak
%----------------------------------------------------------------------------------------
%	LANGUAGES SECTION
%----------------------------------------------------------------------------------------

	
\section{Sprachen}

\cvitem{Deutsch}{Muttersprache}

\cvitem{Englisch}{C1 (fließend)}
\cvitem{Latein}{Latinum}


\end{multicols}



%----------------------------------------------------------------------------------------
%	DATUM, ORT & UNTERSCHRIFT
%----------------------------------------------------------------------------------------
%\vfill
\begin{tabular}{p{3.5cm}}
	\\
	\includegraphics[height=1.5\baselineskip]{pictures/Unterschrift 2025}\\
	\hline \\ \\
\end{tabular}
\personallocation, den \mydate 

%\hspace*{2cm}Marvin März

\end{document}